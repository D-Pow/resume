\documentclass[10pt, letterpaper]{article}

\usepackage[utf8]{inputenc}     % ensure UTF-8 encoding is used
\usepackage[letterpaper, margin=0.5in]{geometry} % document margins, `showframe` will display margins
\usepackage[english]{babel}
\usepackage{times}              % Times New Roman font-family
\usepackage{microtype}          % prevent LaTeX from fusing characters into ligatures, e.g. "fi" becoming one character
% \usepackage[compact]{titlesec}
\usepackage{multicol}           % multicolumns usage: \begin{multicols}{#} ... \end{multicols}
                                % minipage for declaring column widths, multicols for equal widths
\usepackage{float}              % prevent tables from being moved outside their section using [H]
\usepackage{enumitem}           % custom itemize/enumerate styles
\usepackage[hidelinks, pdfpagemode=UseNone, pdfstartview=FitH]{hyperref}
                                    % Create bookmarks, hide link outlines, start with bookmark
                                    % pane closed
                                    % make default PDF-opening view fit full width
\usepackage{makecell}           % multiline table cells
\usepackage{ifthen}             % boolean expressions
\usepackage{tikz}               % \foreach command for looping through args


%-------------------Formatting-------------------%

% Command             10pt    11pt    12pt
% \tiny               5       6       6
% \scriptsize         7       8       8
% \footnotesize       8       9       10
% \small              9       10      10.95
% \normalsize         10      10.95   12
% \large              12      12      14.4
% \Large              14.4    14.4    17.28
% \LARGE              17.28   17.28   20.74
% \huge               20.74   20.74   24.88
% \Huge               24.88   24.88   24.88

%---------------Section Indentation--------------%

% Change what is shown in table of contents with:
% \setcounter{tocdepth}{n}
% \setcounter{secnumdepth}{n}

% Command                            N
% \part{part}                       -1
% \chapter{chapter}                  0
% \section{section}                  1
% \subsection{subsection}            2
% \subsubsection{subsubsection}      3
% \paragraph{paragraph}              4
% \subparagraph{subparagraph}        5



% Remove page numbering from all pages
\pagenumbering{gobble}

% Pages use \flushbottom by default to force all pages to have the same vertical ending location.
% This results in pages that don't fill the entire vertical space on the page to spread their content
% evenly to fill it instead of spacing content normally
\raggedbottom                   % allow page content to end wherever it wants

% Only disable ligatures for "f" so that -- can be dash, double quotes can be a single character, etc.
\DisableLigatures[f]{encoding=*, family=*}

% \setlength{\parskip}{0em}            % space between paragraphs and \\

% Change default \(sub(sub))section styling.
% Define new functions since \usepackage{titlesec} doesn't support \tableofcontents (see: https://tex.stackexchange.com/a/56025/164369).
% This unfortunately removes the ability to see the sections in "Bookmarks" tab of PDF readers, but it's only one page
% so this is acceptable
%
% Attempts were made, but exact formatting for what I wanted could not be obtained.
% These are the titlesec notes taken during attempts
%
% \titleformat{command}{font alterations}{display}{not sure}{before-text display, e.g. numbering}
%
% \titleformat{\section}{\normalfont\Large\bfseries}{
%     \noindent \textbf{\Large \thesection}
%     \hr
% }{1em}{}
%
% \titlespacing{command}{left spacing}{before spacing}{after spacing}[right]
% \titleformat{\section}{\Large}{\thesection}{*0}{}
% \titlespacing*{\section}{0pt}{*0}{*0}


\newcommand{\ressection}[2][12.5]{
    \vspace{#1pt} % default to adding 10pt of space above section title
    \noindent {\Large \ #2}
    \vspace{0pt}
    \hr
}

\newcommand{\ressubsection}[4][]{
    % \parbox is the same as minipage, just with simpler syntax
    % that doesn't allow for some commands/styles to be used inside it.
    % \ragged(side) floats text to that side. Use \centering for centered text
    \vspace{8pt}
    \noindent
    \parbox[c]{.8\textwidth}{
        % large, underlined title with languages in parentheses
        % insert extra space (&nbsp;) to slightly indent title
        \raggedright \ \underline{\large #2} \ (#3) \
        \ifthenelse
            {\equal{#1}{}} % add <URL> if optional #1 != null
            {}
            {\small{$<$\url{#1}$>$}}
    }
    % total width of both parboxes needs to be < 1
    \parbox[c]{.19\textwidth}{
        % dates floated to the right
        \raggedleft (#4)
    }
}


% Remove extra space between \underline and text
\let\oldunderline\underline
\renewcommand{\underline}[1]{\oldunderline{\smash{#1}}}


% make URLs font and size the same as regular text
\urlstyle{same}
% Overwrite old \url command to underline it (since `hidelinks` is active in hyperref)
\let\oldurl\url
\renewcommand{\url}[1]{\oldunderline{\oldurl{#1}}}


% enumitem package:
% Global:
%   Remove separation between lists and the outside world
%   Remove separation between items within lists
%   Make each level of indention equal to what indenting a paragraph
%     would be (0 for first level, 1 par indent for second level, etc)
\setlist[itemize]{nosep, noitemsep, leftmargin=\parindent, label={\large\textbullet}}
% Local:
%   Make second-level bullet style circles instead of dashes
\setlist[itemize,2]{label={\large$\circ$}}

%-----------------------COMMANDS-----------------------%
%\newcommand{name}[number of args][default for optional]{definition}

\newcommand{\email}[1]{
    \underline{\href{mailto:#1}{#1}}
}

\newcommand{\tab}[3][|r|l|]{
    % [1] = Border style/format of table columns
    %  2  = Title
    %  3  = Table content
    {   % extra bracket in order to group the \centering content
        \begin{table}[H] %[H] means put table here, not where it fits best
        \caption*{\textbf{#2}}
        \vspace{-0.3cm} % remove extra space after caption
        \centering
        \begin{tabular}{#1}%
        \hline
        #3
        \end{tabular}
        \end{table}
    }
}

% Use minipage to make multi-column entries of specified widths.
% Usage:
%   \colswidthsof{
%       .3/{
%           Column 1 content
%       },
%       .7/{
%           Column 2 content
%       }
%   }
\newcommand{\colswidthsof}[1]{
    \foreach \w/\x in {#1} {
        \begin{minipage}[t]{\w\textwidth}
        \setlength{\parskip}{5cm}
        \centering
        \x
        \end{minipage}
    }
}

% Attempts at making \colswidthsof more user friendly by extracting widths to separate parameter
% \newcommand{\colsattempt}[2][]{
    % TODO find out how to get items from optional #1 array
    % \foreach \z in {#1} {
    % \z \y \\ % requires newline b/c minipage requires newline to make new rows

    % \foreach \w/\x in {#1/2} {
    %     \begin{minipage}[c]{.3\linewidth}
    %     \centering
    %     \x
    %     \end{minipage}
    % }

    % \foreach \w in {#1} {
    %     \begin{minipage}[c]{\w\linewidth}
    %     \centering

    %     \foreach \x in {#2} {
    %         \w \x
    %     }

    %     \end{minipage}
    % }

    % \newforeach[expand list] \x in {#1,2} {
    %     \begin{minipage}[c]{0.3\linewidth}
    %     \centering

    %     \x

    %     \end{minipage}
    % }
% }

% Usage:
% \colsattempt[.3,.3,.4]{
%     {
%         Column 1 content
%     },
%     {
%         Column 2 content
%     },
%     {
%         Column 3 content
%     }
% }

% Unordered lists
\newcommand{\ul}[1]{%
    \begin{itemize}%
    #1%
    \end{itemize}%
}

% Ordered lists
\newcommand{\ol}[1]{%
    \begin{enumerate}%
    #1%
    \end{enumerate}%
}

% Quotes
\newcommand{\q}[1]{``#1''}

% Tilde
\def\t{$\sim$}

% Horizontal line spreading the page width within margins
\def\hr{
    % Per https://tex.stackexchange.com/questions/421791/preventing-letters-with-descenders-g-q-y-from-affecting-line-spacing
    % remove extra vspace being added due to characters with descenders,
    % e.g. "g", "y", "p", etc. that go below the normal bottom line of text.
    % Other possible way of doing this: \noindent\rule{\textwidth}{.5pt}
    \par\kern-\prevdepth\vspace{3pt}
    \hrule
    \vspace{2pt}
}

% Link to parts of this article tagged with \label{name}
\newcommand{\secref}[1]{\underline{\nameref{#1}}}





\begin{document}

\vspace{-1\multicolsep}
\begin{multicols}{3}
\noindent
\centering

\email{djp460@nyu.edu} \\
(+1) 321-356-5260

\columnbreak

\textbf{\LARGE Devon Powell} \\
\url{https://devon.is-a.dev/}

\columnbreak

\url{https://github.com/D-Pow} \\
\url{https://linkedin.com/in/devon-powell/}
\end{multicols}
\vspace{-1\multicolsep}



\ressection[6]{Skills}

\noindent Languages (ranked 5 $>$ 1)
\par
JavaScript (5), \ Java (4), \ HTML (3), \ CSS (3), \ Python (3), \ SQL (2), \ PHP (1), \ LaTeX (1).

\noindent Technical Skills
\par
Git, \ React, \ MobX, \ Spring, \ Regex, \ Webpack, \ Bash, \ Bootstrap, \ Jest, \ Gradle, \ JUnit, \ Maven, \ SQLite, \ Oracle DB, \ Linux,
\par
Agile (Scrum), \ BitBucket, \ Jira, \ CI/CD (Jenkins, Travis), \ GitHub, \ Heroku, \ Docker, \ Google Cloud, \ Automation Testing.



\ressection{Employment}

\ressubsection{E-Trade Full-Stack Software Engineer - Senior Analyst}{JavaScript, Java, Python}{July 2018 -- Present}

\ul{
    \item \textbf{Mediated company-wide network issues} in personal time via \underline{MockRequests} (see Open-source below).

    \ul{
        \item \textbf{Removed largest front-end blocker} in the company: server rolls occurring twice per week that disable all API endpoints.

        \item Used across many teams and projects; \textbf{improved developer productivity by 40\%}.

        \item Allowed any data-driven JavaScript development to continue unfazed regardless of environment conditions.
    }

    \item Built a pre-market dashboard for analyzing internal systems/servers before markets open (July -- Dec 2018) -- Python/jQuery.

    \ul{
        \item Proposed \textbf{new architectural direction} to enhance data flow, improve readability, and reduce technical debt.

        \item \textbf{Took ownership} of architecture design, implementation, and successful production rollout.
    }

    \item Build 2 platforms for creating mutual fund investment plans -- React SPA/MobX/MVVM/Spring/Oracle DB.

    \ul{
        \item \q{Prebuilt Portfolios} that brought in \textbf{\$2.5 million in its first year} of release.

        \item \q{Automatic Investing} that resulted in a \textbf{37\% growth} of mutual fund purchases and \textbf{15\% growth} in new customers.

        \item \textbf{Improved client performance by 50\%} through restructuring state management and render logic.

        \item Refactored legacy React components to be highly reusable, customizable, and specific to individual use cases.

        \item Enhanced server endpoints by updating data aggregation logic, simplifying client-server communications.
    }

    \item \textbf{Foster a culture of learning and growth} in the organization's developer community.

    \ul{
        \item \textbf{Primary mentor for onboarding all new hires}, including meetings for teaching git, libraries, and development stack.

        \item \textbf{Optimize team productivity} by writing/sharing scripts for automating tasks, including allowing use of local IDEs to code on remote servers, remote file exchanges, Jira ticket autocompletion in git, cookie extraction/refresh during development, etc.

        \item \textbf{Inspire deeper collaboration} across various Slack channels, leading to improvements in coding practices.
    }
}

\ressubsection[play.google.com/store/apps/details?id=com.etrade.mobilepro.activity]{E-Trade Internship}{Java/Android}{June -- Aug. 2017}

\ul{
    \item Developed a panel widget for Edge devices that displayed a user's watch lists, each containing its own set of stocks.

    \item Involved UI/UX design, authentication, state management, caching, and testing (JUnit).

    \item \textbf{Took responsibility for a seamless hand-off} by writing extensive documentation and applying good source control practices.
}



\ressection{Open-source}

\ressubsection[https://www.npmjs.com/package/mock-requests]{MockRequests}{JavaScript}{Sept. 2019}
% \phantomsection \label{MockRequests}
\ul{
    \item Highly user-friendly JavaScript network-mocking library, with features \textbf{most other solutions don't provide}.

    \item Provides a simple, one-stop-shop configuration, with \textbf{no need to change source code} or run local servers.

    \item Features: dynamic mocks based on payload/query parameters, response delays, and cross-browser functionality.
}



\ressection{Projects}

\ressubsection[https://github.com/D-Pow/anime-atsume]{Anime Atsume}{Java, JavaScript}{June -- July 2020}

\ul{
    \item Anime search engine providing the \textbf{quickest possible path} from searching titles to watching episodes, unlike most other sites.

    \item Full-stack: \textbf{SPA React} front-end, \textbf{Spring Boot} back-end, \textbf{Docker} container, \textbf{Heroku} host.
}

\ressubsection[www.atomsofconfusion.com]{Research -- Atoms of Confusion}{C, PHP, HTML/CSS}{Aug. 2016 -- May 2018}

\ul{
    \item Determined code patterns that are objectively confusing, with implications for \textbf{improving design principles and specifications}.
}



\ressection{Education}

\vspace{-1\multicolsep} % remove extra space multicols introduces before its environment
\begin{multicols}{2}
\noindent
\raggedright

\textbf{NYU Tandon School of Engineering} \ -- \ GPA: 3.8/4.0
\underline{Master of Science in Computer Science} (May 2018) Brooklyn, NY

\columnbreak

\textbf{New College of Florida, the State's Honors College}
\underline{Bachelor of Arts in Chemistry} (May 2016) Sarasota, FL

\end{multicols}
\vspace{-1\multicolsep} % remove extra space multicols introduces after its environment


\ressection{Additional Relevant Activities}

\vspace{-1\multicolsep}
\begin{multicols}{2}
\noindent

\ul{
    \item \textbf{Eagle Scout} (2011): Highest rank, top 2\% of all Boy Scouts.
    \item \textbf{Cyber Security Club} (Sept. 2016 -- May 2017).
}

\columnbreak

\ul{
    \item \textbf{NYU Healthcare Hackathon Winner} (Nov. 2016).
    \item \textbf{Japanese} (2019 -- Present): Approaching N5 fluency.
}
\end{multicols}
\vspace{-1\multicolsep}

\end{document}
